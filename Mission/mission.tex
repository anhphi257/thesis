\pagestyle{plain}
\chap{MỞ ĐẦU}

Trong thập kỷ trước, sự phát triển vượt bậc của máy tính cùng với việc giảm giá thành của các thiết bị tính toán đã được đoán trước. Ngày nay, không chỉ các công ty, tập đoàn lớn mà ngay cả cá nhân cũng có thể sở hữu một chiếc máy tính. Để làm việc thuận lợi và hiệu quả, các máy tính được liên kết với nhau sử dụng hệ thống mạng (network). Một trong những hệ thống mạng lớn nhất hiện nay là Internet cho phép một người sử dụng máy tính để trao đổi các thông điệp với các máy tính khác trên Internet. Người dùng làm việc trên Internet được hưởng lợi từ rất nhiều các ứng dụng thuận tiện như World Wide Web (WWW) và e-mail. Tuy nhiên, việc kết nối mở cũng tiềm ẩn những nguy cơ. Xâm nhập, tấn công mạng luôn là mối đe dọa thường trực tới tài sản, uy tính thậm chí là tính mạng tới các cá nhân, tổ chức, doanh nghiệp. Vì vậy việc phát hiện và ngăn chặn cách cuộc tấn công, xâm nhập mạng luôn là mối ưu tiên hàng đầu trong lĩnh vực anh toàn thông tin ngày nay.

\indent Có rất nhiều các phương pháp đã được các nhà khoa học trên thế giới nghiên cứu và ứng dụng vào thực tế giúp phát hiện nhanh chóng và chính xác các cuộc tấn công, xâm nhập mạng. Một trong những phương pháp phổ biến và mạnh mẽ nhất là phương pháp phát hiện xâm nhập dựa trên học máy. Học máy không phải là lĩnh vực mới mẻ mà đã được nghiên cứu từ thế kỷ trước. Tuy nhiên, trong những năm gần đây, sự phát triển của các công nghệ tính toán khiến cho học máy có những thành tựu to lớn. Phương pháp phát hiện xâm nhập mạng dựa trên học máy có thể học dữ liệu từ các cuộc tấn công đã biết, từ đó dự đoán các cuộc tấn công mới. Trong phạm vi kiến thức, đồ án sẽ tập trung nghiên cứu và xây dựng mô hình phát hiện xâm nhập mạng dựa trên học máy, cụ thể với thuật toán Boosted Tree. \\

\indent Đồ án được chia 3 chương với nội dung như sau:
\begin{itemize}
    \item \textbf{Chương 1: Tổng quan về phát hiện xâm nhập mạng}\\ 
    Chương này sẽ giới thiệu tổng quan về các vấn đề an toàn thông tin mạng, các biện pháp phòng chống xâm nhập cũng như các kỹ thuật phát hiện xâm nhập mạng.
    \item \textbf{Chương 2: Mô hình Boosted Tree}\\
    Giới thiệu về phương pháp học máy có giám sát, tổng quan về phương pháp học Ensemble Learning và một mô hình cụ thể của phương pháp này là Boosted Tree.
    \item\textbf{ Chương 3: Mô hình Boosted Tree cho phát hiện xâm nhập mạng}\\
    Áp dụng mô hình đã nghiên cứu vào tập dữ liệu UNSW-NB15.
    \item \textbf{Chương 4: Tổng kết}\\
    Tổng kết bài toán, tóm tắt những kết quả đã đạt được và còn chưa đạt được. Từ đó đề xuất mục tiêu hướng tới cũng như hướng nghiên cứu, phát triển tiếp theo.
\end{itemize}