\pagestyle{plain}
\renewcommand{\bibname}{\Large{TÀI LIỆU THAM KHẢO}}
\begin{thebibliography}{99}
\addcontentsline{toc}{chapter}{TÀI LIỆU THAM KHẢO}
\bibitem{1} Tood McGuiness. Defense In Depth. \textit{http://www.sans.org/rr/securitybasics/defense.php}, November 2001.
\bibitem{2} Patcha, Animesh, and Jung-Min Park. "An overview of anomaly detection techniques: Existing solutions and latest technological trends." \textit{Computer networks} 51, no. 12 (2007): 3448-3470.
% \bibitem{3} \textit{https://en.wikipedia.org/wiki/Bias\%E2\%80\%93variance\_tradeoff}
\bibitem{4} Xgboost Documents. \textit{https://github.com/dmlc/xgboost}
\bibitem{5}  Kline, M. (1990). Mathematical Thought from Ancient to Modern Times. New York: Oxford University Press. pp. 35–37. ISBN 0-19-506135-7.
\bibitem{7} T. Dietterich, Ensemble methods in machine learning, in the first International Workshop on Multiple Classifier Systems, Springer, pp. 1-15, 2000.
\bibitem{8} Stavroulakis P, Stamp M. Handbook of information and communication security.-New York: Springer-Verlag; 2010.
\bibitem{9} Couture M. Real time intrusion prediction based on optimized alerts with hidden Markov model. Journal of Networks 2012;7:311–21.
\bibitem{10} Fragkiadakis AG, Tragos EZ, Tryfonas T, Askoxylakis IG. Design and performance evaluation of a lightweight wireless early warning intrusion detection proto- type. EURASIP Journal on Wireless Communications and Networking 2012;73: 1–18.
\bibitem{11} Kantzavelou I, Katsikas S. A game-based intrusion detection mechanism to confront internal attackers. Computers &amp; Security 2010;29:859–74.
\bibitem{12} Sabahi F, Movaghar A, Intrusion detection: a survey, In: Third international conference on system and network communication, Sliema, Malta, 2008, pp. 23–26.
\bibitem{13}L Li, Zhang G, Nie J, Niu Y, Yao A, The application of genetic algorithm to intrusion detection in MP2P network. In: Third international conference on advances in swarm intelligence, Shenzhen, China, 2012, pp. 390–397.
\bibitem{14} Li Y, Xia J, Zhang S, Yan J, Ai X, Dai K. An efficient intrusion detection system based on support vector machines and gradually feature removal method. Expert Systems with Applications 2012;39:424–30.
\bibitem{15} Kaggle: Your Home for Data Science. \textit{https://www.kaggle.com/}
\bibitem{16} L. Breiman, Random Forests, Machine Learning. 45(1) (2001), 5-32.
\bibitem{numpy} Numpy. \textit{http://www.numpy.org/}
\bibitem{scipy} SciPy.org. \textit{https://www.scipy.org/}
\bibitem{sklearn} scikit-learn: machine learning in Python. \textit{http://scikit\-learn.org/}
\bibitem{missing} Chen, T. and Guestrin, C. (2016). Xgboost: A scalable tree boosting system. In \textit{Proceedings of the 22Nd ACM SIGKDD International Conference on Knowledge Discovery and Data Mining}, KDD ’16, pages 785–794, New York, NY, USA. ACM.
\bibitem{kdd99} KDDCup1999.Available-on: http://kdd.ics.uci.edu/databases/kddcup99/KDDCUP99.html, 2007
\bibitem{nslkdd} NSLKDD. Available on: http://nsl.cs.unb.ca/NSLKDD/, 2009
\bibitem{17} P.Gogoi et al, ”Packet and flow based network intrusion dataset."Contemporary Computing”. Springer Berlin Heidelberg, 2012. P 322-334.
\bibitem{18} McHugh, John, ”Testing intrusion detection systems: a critique of the 1998 and 1999 DARPA intrusion detection system evaluations as performed by Lincoln Laboratory”. ACM transactions on Information and system Security, 3, 2000, p 262-294.
\bibitem{19} V.Mahoney, and K.Philip, “An analysis of the 1999 DARPA/Lincoln Laboratory evaluation data for network anomaly detection."Recent Advances in Intrusion Detection”. Springer Berlin Heidelberg, 2003.
\bibitem{20} A.Vasudevan, E. Harshini, and S. Selvakumar, "SSENet-2011: a network intrusion detection system dataset and its comparison with KDD CUP 99 dataset”, Internet (AH-ICI), 2011, Second Asian Himalayas International Conference on. IEEE.
\bibitem{ixia} PerfectStorm | Ixia \textit{http://www.ixiacom.com/products/perfectstorm}
\bibitem{cve}CVE - Common Vulnerabilities and Exposures (CVE) \textit{ https://cve.mitre.org/}
\bibitem{confusion} Confusion matrix; scikit-learn 0.19.1 documentation \textit{http://scikit-learn.org/stable/auto\_examples/model\_selection/plot\_confusion\_matrix.html}
% \bibitem{knn}  Altman, N. S. (1992). "An introduction to kernel and nearest-neighbor nonparametric regression". The American Statistician. 46 (3): 175–185. doi:10.1080/00031305.1992.10475879
\bibitem{ada} Freund, Yoav; Schapire, Robert E (1997). "A decision-theoretic generalization of on-line learning and an application to boosting". Journal of Computer and System Sciences. 55: 119. CiteSeerX 10.1.1.32.8918 Freely accessible. doi:10.1006/jcss.1997.1504: original paper of Yoav Freund and Robert E.Schapire where AdaBoost is first introduced.
\bibitem{tradeoff}  Bias–variance decomposition, In Encyclopedia of Machine Learning. Eds. Claude Sammut, Geoffrey I. Webb. Springer 2011. pp. 100-101
\bibitem{naivebayes} Domingos, Pedro; Pazzani, Michael (1997). "On the optimality of the simple Bayesian classifier under zero-one loss". Machine Learning. 29: 103–137.
\end{thebibliography}