\pagestyle{plain}
% \renewcommand{\bibname}{\Large{Tổng kết}}
\chapter{TỔNG KẾT}
% \addcontentsline{toc}{chapter}{Tổng kết}

Phát hiện xâm nhập mạng sử dụng học máy tuy không mới nhưng đây là một lĩnh vực nghiên cứu rất tiềm năng. Trong phạm vi của mình, đồ án đã đạt được những kết quả sau:

\begin{itemize}
    \item Trình bày tổng quan về vấn đề xâm nhập mạng, các phương pháp về phát hiện xâm nhập mạng hiện nay.
    \item Nghiên cứu về giải thuật Học kết hợp (Ensemble Learning) cụ thể là thuật toán Boosted Tree. Đây là một thuật toán mạnh mẽ, có khả năng áp dụng cao vào thực tế.
    \item Áp dụng mô hình Boosted Tree vào phát hiện xâm nhập mạng. Từ đó rút ra được những điểm mạnh và điểm yếu của mô hình.
\end{itemize}

Bên cạnh những điều đã nghiên cứu được, đồ án vẫn còn một số hạn chế và định hướng phát triển tiếp theo:
\begin{itemize}
    \item Cải thiện khả năng dự đoán các nhãn tấn công, tránh nhầm lẫn giữa các nhãn với nhau
    \item Thực hiện tiền xử lý dữ liệu. Hiện tại đồ án chưa có bước tiền xử lý dữ liệu từ gói tin thô sang bộ các đặc trưng.
    \item Từ mô hình học được, nhúng vào các hệ thống phát hiện xâm nhập mạng.
\end{itemize}